%------------------------------------
% Dario Taraborelli
% Typesetting your academic CV in LaTeX
%
% URL: http://nitens.org/taraborelli/cvtex
% DISCLAIMER: This template is provided for free and without any guarantee 
% that it will correctly compile on your system if you have a non-standard  
% configuration.
% Some rights reserved: http://creativecommons.org/licenses/by-sa/3.0/
%------------------------------------

%!TEX TS-program = xelatex
%!TEX encoding = UTF-8 Unicode

\documentclass[10pt]{article}
\usepackage{fontspec} 

% DOCUMENT LAYOUT
\usepackage{geometry} 
\geometry{textwidth=5.2in, textheight=8.5in, marginparsep=7pt, marginparwidth=.6in}
\setlength\parindent{0in}

% FONTS
\usepackage[usenames,dvipsnames]{color}
\usepackage{xunicode}
\usepackage{xltxtra}
\defaultfontfeatures{Mapping=tex-text}
\setromanfont [Ligatures={Common}, Numbers={OldStyle}, Variant=01]{Linux Libertine O}%{Times New Roman}
\setmonofont[Scale=0.8]{Linux Libertine O}%Times New Roman}

% ---- CUSTOM COMMANDS
\chardef\&="E050
\newcommand{\html}[1]{\href{#1}{\scriptsize\textsc{[html]}}}
\newcommand{\pdf}[1]{\href{#1}{\scriptsize\textsc{[pdf]}}}
\newcommand{\doi}[1]{\href{#1}{\scriptsize\textsc{[doi]}}}
% ---- MARGIN YEARS
\usepackage{marginnote}
\newcommand{\amper{}}{\chardef\amper="E0BD }
\newcommand{\years}[1]{\marginnote{\scriptsize #1}}
\renewcommand*{\raggedleftmarginnote}{}
\setlength{\marginparsep}{7pt}
\reversemarginpar

% HEADINGS
\usepackage{sectsty} 
\usepackage[normalem]{ulem} 
\sectionfont{\mdseries\upshape\Large}
\subsectionfont{\mdseries\scshape\normalsize} 
\subsubsectionfont{\mdseries\upshape\large} 

% PDF SETUP
% ---- FILL IN HERE THE DOC TITLE AND AUTHOR
\usepackage[dvipdfm, bookmarks, colorlinks, breaklinks, 
% ---- FILL IN HERE THE TITLE AND AUTHOR
	pdftitle={Deren Eaton - vita},
	pdfauthor={Deren Eaton},
	pdfproducer={http://nitens.org/taraborelli/cvtex}
]{hyperref}  
\hypersetup{linkcolor=blue,citecolor=blue,filecolor=black,urlcolor=MidnightBlue} 

% DOCUMENT
\begin{document}
{\LARGE Deren A. R. Eaton}\\[.7cm]
 Yale University \\
21 Sachem St. Office 366\\
New Haven, CT 06511, U.S.A.\\[.2cm]
Phone: \texttt{715-864-9358}\\
Email: \href{mailto:deren.eaton@yale.edu}{deren.eaton@yale.edu}\\
\textsc{url}: \href{http://www.dereneaton.com}{http://www.dereneaton.com}\\

%[0.25cm]
%\vfill
% Born:  March 12, 1879---Ulm, Germany\\
%Nationality:  German/American

%%\hrule
\section*{Current position}
\emph{Postdoctoral Researcher}, Yale University\\
Adviser: Michael Donoghue\\

%\section*{Areas of specialization}
% Evolution • Ecology • Bioinformatics • Botany

%%\hrule
\section*{Appointments held}
\noindent
\years{2004-2005}Research Technician, University of Minnesota\\
\years{2005-2006}Research Assistant, University of Minnesota\\
\years{2006-2007}Undergraduate Teaching Assistant, University of Minnesota\\
\years{2007-2008}Research Fellow, Brown University\\
\years{2008-2013}Graduate Teaching Assistant, University of Chicago\\
\years{2014-2016}Postdoctoral Researcher, Yale University\\

%\hrule
\section*{Education}
\noindent
\years{2007}\textsc{BSc} in Plant Biology, University of Minnesota \\
\years{2007}\textsc{BSc} in Ecology, Evolution and Behavior, University of Minnesota\\
-- Thesis adviser: Jeannine Cavender-Bares\\
\years{2013}\textsc{PhD} in Evolutionary Biology, University of Chicago\\
-- Thesis adviser: Richard Ree\\

%\hrule
\section*{Fellowships, Grants and awards}
\subsection*{Fellowships}
\noindent
\years{2010}NSF East Asian Pacific Summer Institutes Fellowship\\
\years{2011}Lester Armour Graduate Student Fellowship, Field Museum\\
\subsection*{Grants}
\noindent
\years{2009}Pritzker Laboratory Grant, Field Museum\\
\years{2011}Hinds Fund Grant, University of Chicago\\
\years{2011}NSF Doctoral Dissertation Improvement Grant\\
\years{------}NSF DEB Invited Full Proposal (co-PI) \emph{Currently In Review}\\
\subsection*{Awards}
\noindent
\years{2004-2007}Soil, Water, Climate Department scholarship, U Minnesota\\
\years{2004-2006}College of Agriculture, Food and Environmental Sciences scholarship, U Minnesota\\
\years{2005-2007}Dean's list scholarship, U Minnesota\\
\years{2007}Ernst Abbe Award for Plant Biology Majors, U Minnesota\\
\years{2007}Honors, \emph{summa cum laude}, U Minnesota\\
\years{2008}NSF Graduate Research Fellowship, honorable mention\\
\years{2012}Micromorph workshop travel award, Harvard University\\
\years{2015}Postdoctoral Scholars Travel Award, Yale University\\


\section*{Publications and talks}

\subsection*{Journal articles}
\noindent
\years{2012}{\bf Eaton, D.A.R.}, Fenster, C.B., Hereford, J., Huang, S-Q. and R.H. Ree (2012). ``Floral diversity and community structure in \emph{Pedicularis} (Orobanchaceae)'', \emph{Ecology}, 93: S182-S194.\\
\years{2013}Fournier-Level A., Wilczek, A.M., Cooper, M.D., Roe, J.L, Anderson, J.A., {\bf Eaton, D.A.R.}, Moyers, B.T., Petipas, R.H., Schaeffer, R.N., Pieper, B., Reymond, M., Koorneef, M., Welch, S.M., Remington, D.L. and J.S. Schmitt (2013). ``Paths to selection on life-history loci in different natural environments across the native range of \emph{Arabidopsis thaliana}''. \emph{Molecular Ecology}, 22: 3552-3566.\\
\years{2013}Wang, X., Zhao, L., {\bf Eaton, D.A.R.} and Z. Guo (2013). ``Identification of SNP markers for inferring phylogeny in temperate Bamboos (Poaceae: Bambusoideae) using RAD tag sequencing''. \emph{Molecular Ecology Resources}, 13: 938-945.\\
\years{2013}{\bf Eaton, D.A.R.} and R.H. Ree (2013). ``Inferring Phylogeny and Introgression using genomic RADseq Data: An Example from Flowering Plants (\emph{Pedicularis}: Orobanchaceae)''. \emph{Systematic Biology}, 62: 689-706.\\
\years{2014}Hipp, A., {\bf Eaton, D.A.R.}, Cavender-Bares, J., Fitzek, E., Nipper, R. and P. Manos (2014). ``A framework phylogeny of the New World oak clade based on sequenced RAD data''. \emph{PLoS ONE} 9(4): e93975.\\
\years{2014}{\bf Eaton, D.A.R.} (2014). ``PyRAD: \emph{de novo} Assembly of RAD/GBS data for phylogenetic and introgression analyses''. \emph{Bioinformatics}, 30(13): 1844-1849.\\
\years{2014}{\bf Eaton, D.A.R.} (2014). ``On the Evolutionary Consequences of Interspecific Reproductive Interactions''. \emph{PhD dissertation}.\\
\years{2014}Escudero, M., {\bf Eaton, D.A.R.}, Hahn, M. and A. Hipp (2014). ``Genotyping-by-sequencing as a tool for phylogenetic inference and testing ancestral hybridization: A case study in \emph{Carex} (Cyperaceae)''. \emph{Molecular Phylogenetics and Evolution} 79: 359-367.\\
\years{2015}Cavender-Bares, J., Gonzalez-Rodriguez, A., {\bf Eaton, D.A.R.}, Hipp, A., Buelke, A., and P. Manos (2015). ``Phylogeny and biogeography of the American live oaks (\emph{Quercus} subsection Virentes): A genomic and population genetic approach''. \emph{Molecular Ecology}, 24(14): 3668-3687.\\
\years{2015}{\bf Eaton, D.A.R.}, Hipp, A., Gonzalez-Rodriguez, A. and J. Cavender-Bares (2015). ``Historical introgression among the American live oaks and the comparative nature of tests for introgression''. \emph{Evolution}. Accepted, In Press. 

\subsection*{Journal Articles In Prep}
\noindent
\years{------}{\bf Eaton, D.A.R.}, Huang S-Q. and R.H. Ree (In Prep). ``Interspecific reproductive character displacement promotes intraspecific floral divergence in a widespread alpine plant species''.\\
\years{------}{\bf Eaton, D.A.R.}, Ree, R.H. and S-Q. Huang (In Prep). ``Evolutionary constraint and pollen-pistil relationships above and below the species-level in Chinese \emph{Pedicularis}''.\\
\years{------}{\bf Eaton, D.A.R.}, Spriggs, E., Park, B. and M.J. Donoghue (In Prep). ``Misconceptions on missing data in RADseq phylogenetics with a deep-scale example from flowering plants (\emph{Viburnum}: Adoxaceae)''.\\
\years{------}Forsman, Z.H., Knapp, I.S.S., {\bf Eaton, D.A.R.}, Belcaid, M. and R.J. Toonen (In Prep). ``Identifying loci from metagenomic RAD seq data; whole mitochondrial genomes, ribosomal genes, and other large contigs from the coral holobiont provides insights into the \emph{Porites lobata/compressa} species complex''.\\

\subsection*{Popular writing}
\noindent
\years{2013}Hipp, A.L., Manos, P.S., Cavender-bares, J.C., {\bf Eaton, D.A.R.}, and R. Nipper. ``Using Phylogenomics to infer the evolutionary history of Oaks''. \emph{The Journal of International Oaks}. International Oak Journal, 24: 61-71.\\
\years{2015}{\bf Eaton, D.A.R.} Personal writing/blog posts at \href{http://www.dereneaton.com}{http://www.dereneaton.com}\\


\subsection*{Invited talks}
\years{2012}Shangri-la Alpine Botanic Garden, Yunnan, China, S-Q. Huang Lab\\
\emph{Floral diversity and phylogenetic community ecology in \emph{Pedicularis}}\\
\years{2012}University of Illinois at Chicago -- EEB Department Seminar\\
\emph{Introgression, isolation and interference: reproductive conflict in a diverse clade of flowering plants}\\
\years{2012}Arnold Arboretum, Harvard University -- MicroMorph workshop\\
\emph{Introgression, isolation and interference: reproductive conflict in a diverse clade of flowering plants}\\
\years{2013}Field Museum -- Chicago Plant Sciences Symposium\\
\emph{Isolation, introgression and floral divergence in \emph{Pedicularis} as revealed through genomic RAD sequences''}\\
\years{2013}Yale University -- Donoghue/Near lab Seminar\\
\emph{Evolutionary consequences of interspecific reproductive interactions in \emph{Pedicularis}}.\\
\years{2013}Ernst Mayr Symposium -- Evolution meeting, Snowbird, Utah\\
\emph{Detecting genomic introgression at the phylogenetic scale}\\
\years{2014}Yale University -- EEB Department Seminar\\
\emph{Inferring phylogeny and introgression from genomic RADseq data}\\
\years{2015}Hawaii Institute of Marine Biology -- Department Seminar\\
\emph{Insights from clade-level phylogenomics across evolutionary scales}\\
\years{2015}University of Hawaii at Manoa -- EEB Department Seminar\\
\emph{Insights from clade-level phylogenomics across evolutionary scales}\\
\years{2015}Harvard University -- C. Davis Lab Seminar\\
\emph{Insights from clade-level phylogenomics across evolutionary scales}\\
\years{2015}Universidad Nacional Autonoma de Mexico\\
\emph{An introduction to RADseq Phylogenomics}\\
\years{2015}American Museum of Natural History -- Comparative Biology Seminar\\
\emph{Phylogeny, Introgression, and adaption as revealed through genomic RADseq analyses}\\
\years{2015}University of Massachusetts, Amherst -- EEB Department Seminar\\
\emph{Insights from clade-level phylogenomics across evolutionary scales}\\

\subsection*{Talks at meetings/conferences}
\years{2009}Evolution meeting, Moscow, Idaho\\
\years{2010}Graduate student seminar, University of Chicago\\
\years{2010}Dissertation proposal, University of Chicago\\
\years{2011}Graduate student seminar, University of Chicago\\
\years{2012}Graduate student seminar, University of Chicago\\
\years{2012}Pritzker Laboratory Seminar, Field Museum\\
\years{2012}Evolution meeting, Ottawa, Canada\\
\years{2013}Graduate student seminar, University of Chicago\\
\years{2013}Evolution meeting, Snowbird, Utah\\
\years{2014}Botany meeting, Boise, Idaho\\
\years{2014}Evolution meeting, Raliegh, North Carolina\\
\years{2015}Society for Systematic Biologists stand-alone meeting, Ann Arbor, Michigan\\
\years{2015}Botany meeting, Edmonton, Alberta\\

\section*{Teaching}
\subsection*{University Teaching Experience}
\years{2007}\emph{Introduction to Biochemistry}, University of Minnesota (A. Hooper)\\
Teaching Assistant -- Led discussion groups, study sessions, and grading.\\
\years{2009}\emph{Ecology and Evolution of the Southwest Deserts}, University of Chicago (E. Larsen)\\
Teaching Assistant -- Led discussion groups, study sessions, and grading. \\
\years{2009}\emph{Field course in desert ecology and evolution}, University of Chicago (E. Larsen)\\
Teaching Assistant -- Organized undergraduate student projects on a three week field course to the southwestern US deserts.\\
\years{2009}\emph{Ecology and evolution of southwest deserts}, University of Chicago (E. Larsen)\\
Guest Lecture -- ``plant adaptation and C3/C4 photosynthesis''\\
\years{2011}\emph{Environmental Ecology}, University of Chicago (T. Price)\\
Teaching Assistant -- Led discussion groups, field-related lab classes, and grading. \\
\years{2013}\emph{Ecology and evolution of the Southwest Deserts}, University of Chicago (E. Larsen)\\
Teaching Assistant -- Organized undergraduate student projects on a three week field course to the southwestern US deserts.\\

\subsection*{Workshops taught}
\years{2015}Introduction to RADseq Phylogenomics: \emph{pyrad} workshop. University of Hawaii at Manoa.\\
\years{2015}Introduction to RADseq Phylogenomics: \emph{pyrad} workshop. Universidad Nacional Autonoma de Mexico.\\

%\hrule
\section*{Service to the profession}
\subsection*{Scientific software (github.com/dereneaton)}
\years{2015}\emph{ipyrad} -- Python module for interactive RADseq assembly, genomic analyses, and plotting.\\
\years{2015}\emph{simrrls} -- Program for simulation of RADseq data with realistic locus dropout. \\
\years{2014}\emph{pyrad} -- Program for assembly of RADseq data sets and analyses of introgression. \\

\subsection*{Outreach}
\years{2009-2011} SERTS -- Northwestern University\\
Mentored undergraduates in collection-based research at the Field Museum.\\
\years{2012} Featured Scientist for the DNA Discovery Center -- Field Museum of Natural History.\\
Talked with visitors and answered questions on the Pritzker Lab Facebook page. \\
\years{2009-2013} Evolutionary Discussion Group Organizer -- Field Museum of Natural History\\
Organized an evolution focused journal club for Chicago area researchers.\\ 
\years{2009-2013} PlantingScience -- Botanical Society of America\\
Mentored middle- and high-school students in plant science projects.\\
\years{2013} Undergraduate Research Symposium Judge -- University of Chicago\\
\years{2013} Organized Ecological Speciation graduate reading group -- University of Chicago\\
\years{2008-2015} Member's night -- Field Museum of Natural History\\
Hosted a booth with videos, photos, and specimens to share collecting and herbarium research with Field Museum members behind the scenes.\\
\years{2015} Yale Data Hack -- Team leader and presenter. \\

\subsection*{Professional societies}
Society for Systematic Biologists (SSB)\\
Botanical Society of America (BSA)\\

\subsection*{Volunteered peer-review}
NSF Department of Environmental Biology (DEB)\\
Proceedings of the National Academy of Sciences\\
Systematic Biology\\
PLOS One\\
Ecology\\
Proceedings of the Royal Society: B\\
American Journal of Botany\\
Journal of Biogeography\\
Molecular Ecology\\
Molecular Ecology Resources\\
Genome Research\\
Methods in Ecology and Evolution\\
BMC Research Notes\\
Molecular Phylogenetics and Evolution\\
Bioinformatics\\
Plant species biology\\
Heredity\\

\subsection*{Media/Press coverage}
Yale Forestry Data Hack 2015 -- Competition winner: Team flower hacks [\href{http://epi.yale.edu/the-metric/visualizing-future}{link}].\\
Tale of Two Scientific Fields -- Ecology and Phylogenetics -- Offers New Views of Earth's Biodiversity. [\href{http://nsf.gov/news/news_summ.jsp?cntn_id=125048}{link}].\\
Twitter handle: @dereneaton\\
\vfill

% \subsection*{Workshops (participated)}
% Yale Forestry Data Hack 2015 -- Competition winner: Team flower hacks 
% Tale of Two Scientific Fields -- Ecology and Phylogenetics -- Offers New Views of Earth's Biodiversity.




\begin{center}
{\scriptsize  Last updated: \today\- •\- 
% ---- PLEASE LEAVE THIS BACKLINK FOR ATTRIBUTION AS PER CC-LICENSE
Typeset in \href{http://nitens.org/taraborelli/cvtex}{
\fontspec{Times New Roman}\XeTeX }\\
% ---- FILL IN THE FULL URL TO YOUR CV HERE
\href{http://dereneaton.com/cvtex}{http://dereneaton.com/cvtex}}
\end{center}

\end{document}


%%% Local Variables:
%%% mode: Latex
%%% TeX-PDF-mode: t
%%% TeX-engine: xetex
%%% End: